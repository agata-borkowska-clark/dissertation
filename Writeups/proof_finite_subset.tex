\documentclass[11pt,journal]{IEEEtran}
%\usepackage{hyperref}
%\usepackage[breaklinks]{hyperref}
\usepackage{breakurl}
\usepackage{url}
\usepackage{amsfonts}
\usepackage{amsmath}
\usepackage{amssymb}
\usepackage{listings}
% Set listings to use small monospaced font.
\lstset{basicstyle=\small\ttfamily,tabsize=4}
\usepackage{graphicx}
\ifCLASSOPTIONcompsoc
% IEEE Computer Society needs nocompress option
% requires cite.sty v4.0 or later (November 2003)
\usepackage[nocompress]{cite}

\else
% normal IEEE
\usepackage{cite}
\fi

\hyphenation{op-tical net-works semi-conduc-tor}


\begin{document}	

	\onecolumn
	\appendices

\section{'A subset of a finite set is finite' - Proof}
Let $A$ and $B$ be sets, with $A \subseteq B$ and $B$ finite. Let us define $[n]$ to be the set of all elements of $\mathbb{N}$ less than $n$, i.e.~$[n] = \{0,1,...,n-1\}$.

Since $B$ is finite, by the definition of finiteness there is $n \in \mathbb{N}$ such that there exists a bijection between $B$ and $[n]$. Hence it suffices to prove that any subset of $[n]$ for $n \in \mathbb{N}$ is finite. We proceed by induction.

When $n = 0$, $[n] = \varnothing$, and trivially all subsets of the empty set are finite.

Let $n > 0$ and assume that all subsets of $[n-1]$ are finite. 

Note that $[n] = \{0,1,...,n-1\} = \{0,1,...,n-2\} \cup \{n-1\} = [n-1] \cup \{n-1\}$. Let $x \subseteq [n]$. Then either $n-1 \notin x$ or $n-1\in x$. In the first case, $x \subseteq [n-1]$, and thus it is finite.

Otherwise, $x\backslash\{n-1\} \subseteq [n-1]$ and is finite. Therefore there exists a $k \in \mathbb{N}$ such that there is a bijection $f: x\backslash\{n-1\} \rightarrow [k]$. Then $f' = f \cup \{(n-1, k)\}$ is a bijection $f': x \rightarrow [k+1]$ and by the inductive property of the natural numbers, $k+1 \in \mathbb{N}$.

Hence, any $x \subseteq n$ is finite.

$\square$
	
	% that's all folks
\end{document}

