\documentclass[11pt,journal]{IEEEtran}
%\usepackage{hyperref}
%\usepackage[breaklinks]{hyperref}
\usepackage{breakurl}
\usepackage{url}
\ifCLASSOPTIONcompsoc
% IEEE Computer Society needs nocompress option
% requires cite.sty v4.0 or later (November 2003)
\usepackage[nocompress]{cite}

\else
% normal IEEE
\usepackage{cite}
\fi

\hyphenation{op-tical net-works semi-conduc-tor}


\begin{document}
	\title{Expanding the proof rule base of AtelierB automated prover - Research Proposal}
	
	\author{Agata~Borkowska,~UID: 1690550,~\IEEEmembership{MSc in Computer Science,~University of Warwick}% <-this % stops a space
		\protect\\
		\thanks{}}
	
	% The paper headers
	
	\markboth{}%
	{ \MakeLowercase{\textit{}}: }
	
	\IEEEcompsoctitleabstractindextext{%
		\begin{abstract}
			%\boldmath
			AtelierB is a tool for formal software development through refinement, using the B-method. It incorporates an automated prover, which has been recognized as the most thorough prover for B set theory, and has been used as a basis for many others. Nevertheless it has multiple shortcomings. Various approaches have been suggested and taken to improve its performance, including extensions to the proof rule base, created by the users. In this work we aim to create such an extension, ensuring that all added rules are sound and well-reasoned. We also aim to identify any limitations of this approach. The secondary goal is to improve the robustness of the software without straying from pure B method, and taking into account the ease of use. As a metric of our success, we use the benchmarks proposed by Conchon and Iguernala\cite{survey}.
	\end{abstract}
	\begin{IEEEkeywords}
		B method, formal verification
	\end{IEEEkeywords}}
	% IEEEtran.cls defaults to using nonbold math in the Abstract.
	% This preserves the distinction between vectors and scalars. However,
	% if the journal you are submitting to favors bold math in the abstract,
	% then you can use LaTeX's standard command \boldmath at the very start
	% of the abstract to achieve this. Many IEEE journals frown on math
	% in the abstract anyway. In particular, the Computer Society does
	% not want either math or citations to appear in the abstract.
	
	% Note that keywords are not normally used for peerreview papers.
	
	% make the title area
	\maketitle
	
	
	% To allow for easy dual compilation without having to reenter the
	% abstract/keywords data, the \IEEEcompsoctitleabstractindextext text will
	% not be used in maketitle, but will appear (i.e., to be "transported")
	% here as \IEEEdisplaynotcompsoctitleabstractindextext when compsoc mode
	% is not selected <OR> if conference mode is selected - because compsoc
	% conference papers position the abstract like regular (non-compsoc)
	% papers do!
	\IEEEdisplaynotcompsoctitleabstractindextext
	% \IEEEdisplaynotcompsoctitleabstractindextext has no effect when using
	% compsoc under a non-conference mode.
	
	
	% For peer review papers, you can put extra information on the cover
	% page as needed:
	% \ifCLASSOPTIONpeerreview
	% \begin{center} \bfseries EDICS Category: 3-BBND \end{center}
	% \fi
	%
	% For peerreview papers, this IEEEtran command inserts a page break and
	% creates the second title. It will be ignored for other modes.
	\IEEEpeerreviewmaketitle
	
	
	
	\section{Introduction}
	
	\section{Related Works}
	
	\section{Project Aims}
	
	\subsection{Choice of Scenarios}
	
	\subsection{Metrics}
	
	\subsection{Expected learning outcomes}
	
	\subsection{Appropriateness of Research Methods}
	
	\section{Project Management}
	
	\subsection{Methodology}
	
	\subsection{Timeline}
	Key dates, as listed by the CS907 Dissertation Project website, are:
	
	\begin{itemize}
		\item \textbf{19\textsuperscript{th} January:} Registration of dissertation topics
		\item \textbf{16\textsuperscript{th} February:} Submission of project proposals
		\item \textbf{24-28\textsuperscript{th} April:} Project presentations
		\item \textbf{6\textsuperscript{th} July:} Submission of interim reports
		\item \textbf{14\textsuperscript{th} September:} Submission of dissertation
	\end{itemize}

	It is also important to take into account dates of terms, which are 9\textsuperscript{th} January to 18\textsuperscript{th}March for the Spring Term, and 24\textsuperscript{th} April to 1\textsuperscript{st} July for the Summer term, with the university examinations commencing on or after the 15\textsuperscript{th} May, and being spread over a period of about two weeks. Therefore, it is to be expected that little progress will be made during Spring Term and especially in May, with the bulk of the work being done over holiday and after the examination period.
		
	\subsection{Progress}
	
	\subsection{Constraints and Risks}
	\subsubsection{Copyrights for AtelierB software}
	
	\subsubsection{Risk of data loss or machine failure}
	A GitHub repository has been set up to contain a remote back up of the work done so far. It has the additional benefits of allowing work from multiple machines, and convenient tracking of changes. The address of the repository is: \url{https://github.com/agata-borkowska/dissertation}~.
	
	\section{Concluding remarks}
	
	\IEEEPARstart{}{} 
	
	\begin{thebibliography}{1}
		\bibitem{survey}
		S.~Conchon and M.~ Iguernlala, "Increasing Proofs Automation Rate	of Atelier-B Thanks to Alt-Ergo" in \emph{Proc.~1st Int.~Conf.~Reliability, Safety and Security of Railway Systems} (RSSRail 2016), Springer, 2016, pp.~243-253
		
		
	\end{thebibliography}
	
	% that's all folks
\end{document}

